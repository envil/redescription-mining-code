\section{Redescription Mining}
\label{sec:redescription-mining}

\note{It might not hurt to have one more example in this chapter, if
  space permits.}
Redescription mining aims at simultaneously finding multiple
descriptions of a subset of entities which is not previously
specified.  This is in contrast with other methods like Emerging
Patterns Mining (EPM), Contrast Set Mining (CSM) and Subgroup
Discovery (SD) (see \cite{kralj09supervised} for a unifying survey) or
general classification methods, where target subsets of entities are
specified via labels.  Currently, redescription mining is a purely
descriptive approach, its predictive power remains to be explored.
Since its introduction in~\cite{ramakrishnan04turning} various
algorithms have been proposed for Boolean redescription mining, based
on approaches including decision
trees~\cite{ramakrishnan04turning,kumar07redescription},
co-clusters~\cite{parida05redescription}, and frequent
itemsets~\cite{gallo08finding}. In~\cite{galbrun11black}, we extended
redescription mining to categorical and numerical variables.

More formally, we consider data that contains entities $E$ with two sets
of characterizing variables, e.g.\ the fauna and the bioclimatic
conditions. 
Boolean variables can be interpreted as a truth value
assignment in a natural way.  For categorical and real-valued
variables, truth value assignments are induced by relations denoted using Iverson bracket $[v=c]$
and $[a \leq v \leq b]$, respectively, where $c$ is some category and
$[a, b]$ an interval.  These truth assignments and their negations
constitute \emph{literals} which can be combined using the Boolean
operators $\land$ (and) and $\lor$ (or) to form \emph{queries}.
The support of a query $q$ is the subset of entities for which
the query holds true, that is 
$\supp(q) = \{e\in E : q\text{ is true for } e\}$.
We refer to the two sets of variables informally as left and right
hand side data, and the queries over them, respectively, as left and
right hand side queries, denoted as  $q_\iLHS$ and  $q_\iRHS$ respectively.
Then, a redescription is simply a pair of queries over variables from the
two sets, $R=(q_\iLHS,q_\iRHS)$.    
Its \emph{accuracy} is 
measured using the \emph{Jaccard coefficient} 
\[
\jacc(R)= \jacc(q_\iLHS,q_\iRHS) = \frac{\abs{\supp(q_\iLHS,q_\iRHS)}}%
{\abs{\supp(q_\iLHS)\cup\supp(q_\iRHS)}}.
\]
We compute a \pValue{} that represents the probability that two random
queries with marginal probabilities (i.e.\ the fraction of entities
supporting them) equal to those of $q_\iLHS$ and $q_\iRHS$ have an
intersection equal to or larger than $\abs{\supp(q_\iLHS,
  q_\iRHS)}$. This probability uses the binomial distribution and is given by \[
\text{pvalM}(q_\iLHS, q_\iRHS) = \sum_{s=\abs{\supp(q_\iLHS, q_\iRHS)}}^{\abs{E}} {\abs{E} \choose s} (p_R)^s (1- p_R)^{\abs{E} - s},\]
where $p_R = \abs{\supp(q_\iLHS)}\abs{\supp(q_\iRHS)}/\abs{E}^2.$
The higher the \pValue, the more likely it is to observe such a
support for independent queries, and the less significant the query.

The task consists in finding significant accurate redescriptions, in other words, pairs of
queries, one query for both sets of variables, such that both queries
describe almost the same set of entities.

When the data is geospatial, that is, the entities are connected to
geographical locations, the task is called \emph{geospatial redescription mining}.
 A meaningful geospatial redescription should define
coherent areas using expressive queries.

\emph{Niche-finding} is a particular instance of geospatial redescription
mining ---and a task of great importance for biologists.  The
bioclimatic constraints that must be met for a certain species to
survive constitute that species' bioclimatic envelope, or
niche~\cite{grinnell17niche}.  Finding such envelopes can help, e.g.\
to predict the results of global warming~\cite{pearson03predicting}.
A number of methods, involving regression, neural networks, and
genetic algorithms (see~\cite{soberon05interpretation}) have been
developed over the past ten years to model the bioclimatic envelope,
\textsc{Maxent}~\cite{phillips2006maximum} and \textsc{BIOMOD}~\cite{thuiller09biomod}, being good examples of a
modelling tools used in this domain, the former providing a graphical user interface while the latter is a text-based tool.  But to the best of our knowledge,
none of these methods allows automatically finding both the set of
species and their envelope.

%%% Local Variables: 
%%% mode: latex
%%% TeX-master: "siren_iid"
%%% End: 
