% THIS IS SIGPROC-SP.TEX - VERSION 3.1
\documentclass{sig-alternate}

% Additional packages
\usepackage{graphicx}
\usepackage{url}
\usepackage{algorithm}
\usepackage[noend]{algpseudocode}
\usepackage{amsmath}
%\usepackage{amsthm}
\usepackage{amsfonts}
\usepackage{amssymb}
\usepackage{booktabs}
\usepackage{multirow}
\usepackage{ifpdf}
% Custom commands
\usepackage{custom_commands}


\begin{document}
\special{papersize=8.5in,11in}
\setlength{\pdfpageheight}{11in}%{\paperheight}
\setlength{\pdfpagewidth}{8.5in}%{\paperwidth}


\title{Siren: An Interactive Tool for Mining Geospatial Redescriptions}
\subtitle{[Demo]}

\numberofauthors{2} 
\author{
% 1st. author
\alignauthor
Esther Galbrun\\
       \affaddr{Department of Computer Science}\\
       \affaddr{University of Helsinki, Finland}\\
       \email{galbrun@cs.helsinki.fi}
% 2nd. author
\alignauthor
Pauli Miettinen\\
       \affaddr{Max Plack Institute for Informatics}\\
       \affaddr{Saarbr{\"u}ecken, Germany}\\
       \email{pmiettin@mpi-inf.mpg.de}
}

\maketitle
\begin{abstract}
  We present \textsc{Siren}, an interactive tool for mining geospatial
  redescriptions.  Redescription mining is a powerful data analysis
  tool that aims at finding alternative descriptions the same
  entities.  For example, in biology, an important task is to identify
  the bioclimatic constraints that allow some species to survive, that
  is, to describe geographical regions in terms of both their
  bioclimatic conditions and the fauna that inhabits them.  When the
  entities are geographic locations, we qualify the redescriptions as
  geospatial.  Using \textsc{Siren}, a user can explore data of his
  interest by visualizing geospatial redescriptions on a map,
  interactively edit, expand and filter them.  We will demonstrate our
  system on two data sets, one about European mammals and climate, the
  other containing US census and election funding data.
\end{abstract}

\category{H.2.8}{Information Systems}{Database Applications}[Data Mining]
\category{H.5.2}{Information Interfaces and Presentation}{User Interfaces}

\terms{Redescription Mining, Geospatial data, Application}

%\keywords{ACM proceedings, \LaTeX, text tagging} % NOT required for Proceedings

\section{Introduction}
Finding multiple ways to characterize the same entities is an important way of gaining insight in your data.
In
redescription mining the input contains entities with two sets of
characterizing variables. The task is to find a pair of queries, one
query for both sets of variables, such that both queries describe
(almost) the same set of entities.

In biology, the bioclimatic constraints that must be met 
for a certain species to survive constitute that species' bioclimatic envelope
(or niche\footnote{The term \emph{niche} is in this paper used in Grinnellian
sense \cite{grinnell17niche}, considering only environmental variables, not 
inter-species competition or such.}),
and finding such envelopes can help, e.g.\ to predict the results of 
global warming~\cite{pearson03predicting}.
This problem correspond to f
It is here where \emph{redescription mining} comes to help.  In niche-finding, the entities
would be spatial locations, one set of variables would be the fauna
and the other set would contain the bioclimatic variables. A very
simple example of a redescription in this setting could say that the
area where polar bears live is the area where March's mean temperature
is between $-16$ and $-11$ degrees Celsius and May's mean temperature
is between $-3$ and $-7$ degrees Celsius.




\section{Redescription mining}

\section{Use-case scenarios}
\label{sec:scenarios}

\section{Related work}
\label{sec:related-work}




\section{Conclusions}
This paragraph will end the body of this sample document.
Remember that you might still have Acknowledgments or
Appendices; brief samples of these
follow.  There is still the Bibliography to deal with; and
we will make a disclaimer about that here: with the exception
of the reference to the \LaTeX\ book, the citations in
this paper are to articles which have nothing to
do with the present subject and are used as
examples only.


\bibliographystyle{abbrv}
\nocite{*}
\bibliography{bibsiren}  

\balancecolumns
% That's all folks!
\end{document}

%%% Local Variables: 
%%% mode: latex
%%% TeX-master: t
%%% End: 
