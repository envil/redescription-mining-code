% THIS IS SIGPROC-SP.TEX - VERSION 3.1
\documentclass{sig-alternate}

\begin{document}
\special{papersize=8.5in,11in}
\setlength{\pdfpageheight}{11in}%{\paperheight}
\setlength{\pdfpagewidth}{8.5in}%{\paperwidth}


\title{Siren: A program for visualizing spatial redescription
mining}
\subtitle{[Demo]}
%
% You need the command \numberofauthors to handle the 'placement
% and alignment' of the authors beneath the title.
%
% For aesthetic reasons, we recommend 'three authors at a time'
% i.e. three 'name/affiliation blocks' be placed beneath the title.
%
% NOTE: You are NOT restricted in how many 'rows' of
% "name/affiliations" may appear. We just ask that you restrict
% the number of 'columns' to three.
%
% Because of the available 'opening page real-estate'
% we ask you to refrain from putting more than six authors
% (two rows with three columns) beneath the article title.
% More than six makes the first-page appear very cluttered indeed.
%
% Use the \alignauthor commands to handle the names
% and affiliations for an 'aesthetic maximum' of six authors.
% Add names, affiliations, addresses for
% the seventh etc. author(s) as the argument for the
% \additionalauthors command.
% These 'additional authors' will be output/set for you
% without further effort on your part as the last section in
% the body of your article BEFORE References or any Appendices.

\numberofauthors{2} %  in this sample file, there are a *total*
% of EIGHT authors. SIX appear on the 'first-page' (for formatting
% reasons) and the remaining two appear in the \additionalauthors section.
%
\author{
% You can go ahead and credit any number of authors here,
% e.g. one 'row of three' or two rows (consisting of one row of three
% and a second row of one, two or three).
%
% The command \alignauthor (no curly braces needed) should
% precede each author name, affiliation/snail-mail address and
% e-mail address. Additionally, tag each line of
% affiliation/address with \affaddr, and tag the
% e-mail address with \email.
%
% 1st. author
\alignauthor
Esther Galbrun\\%\titlenote{Dr.~Trovato insisted his name be first.}\\
       \affaddr{Department of Computer Science}\\
       %\affaddr{1932 Wallamaloo Lane}\\
       \affaddr{University of Helsinki, Finland}\\
       \email{galbrun@cs.helsinki.fi}
% 2nd. author
\alignauthor
Pauli Miettinen\\
       \affaddr{Max Plack Institute for Informatics}\\
       \affaddr{Saarbr{\"u}ecken, Germany}\\
       \email{pmiettin@mpi-inf.mpg.de}
}

%\date{30 July 1999}
% Just remember to make sure that the TOTAL number of authors
% is the number that will appear on the first page PLUS the
% number that will appear in the \additionalauthors section.

\maketitle
\begin{abstract}
Redescription mining is COOL!. Redescriptions on spatial data look
COOL! \textsc{Siren} can show redescriptions on maps. COOL!
\end{abstract}

% A category with the (minimum) three required fields
\category{H.4}{Information Systems Applications}{Miscellaneous}
%A category including the fourth, optional field follows...
\category{D.2.8}{Software Engineering}{Metrics}[complexity measures, performance measures]

\terms{Theory}

%\keywords{ACM proceedings, \LaTeX, text tagging} % NOT required for Proceedings

\section{Introduction}


\section{Redescription mining}

\section{Use-case scenarios}
\label{sec:scenarios}

\section{Related work}
\label{sec:related-work}




\section{Conclusions}
This paragraph will end the body of this sample document.
Remember that you might still have Acknowledgments or
Appendices; brief samples of these
follow.  There is still the Bibliography to deal with; and
we will make a disclaimer about that here: with the exception
of the reference to the \LaTeX\ book, the citations in
this paper are to articles which have nothing to
do with the present subject and are used as
examples only.


\bibliographystyle{abbrv}
\bibliography{bibsiren}  

\balancecolumns
% That's all folks!
\end{document}

%%% Local Variables: 
%%% mode: latex
%%% TeX-master: t
%%% End: 
