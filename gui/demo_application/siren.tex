% THIS IS SIGPROC-SP.TEX - VERSION 3.1
\documentclass{sig-alternate}

% Additional packages
\usepackage{graphicx}
\usepackage{color}
\usepackage{url}
\usepackage{algorithm}
\usepackage[noend]{algpseudocode}
\usepackage{amsmath}
%\usepackage{amsthm}
\usepackage{amsfonts}
\usepackage{amssymb}
\usepackage{booktabs}
\usepackage{multirow}
\usepackage{ifpdf}
% Custom commands
%\usepackage{custom_commands}

\newcommand{\note}[1]{{\color{red}#1}}

\newcommand{\Siren}{\textsc{Siren}}
\newcommand{\ReReMi}{\textsc{ReReMi}}

\begin{document}
\special{papersize=8.5in,11in}
\setlength{\pdfpageheight}{11in}%{\paperheight}
\setlength{\pdfpagewidth}{8.5in}%{\paperwidth}


\title{Siren: An Interactive Tool for Mining Geospatial Redescriptions}
\subtitle{[Demo]}

\numberofauthors{2} 
\author{
% 1st. author
\alignauthor
Esther Galbrun\\
       \affaddr{Department of Computer Science}\\
       \affaddr{University of Helsinki, Finland}\\
       \email{galbrun@cs.helsinki.fi}
% 2nd. author
\alignauthor
Pauli Miettinen\\
       \affaddr{Max Plack Institute for Informatics}\\
       \affaddr{Saarbr{\"u}ecken, Germany}\\
       \email{pmiettin@mpi-inf.mpg.de}
}

\maketitle
\begin{abstract}
  We present \Siren, an interactive tool for mining geospatial
  redescriptions.  Redescription mining is a powerful data analysis
  tool that aims at finding alternative descriptions the same
  entities.  For example, in biology, an important task is to identify
  the bioclimatic constraints that allow some species to survive, that
  is, to describe geographical regions in terms of both their
  bioclimatic conditions and the fauna that inhabits them.  When the
  entities are geographic locations, we qualify the redescriptions as
  geospatial.  Using \textsc{Siren}, a user can explore data of his
  interest by visualizing geospatial redescriptions on a map,
  interactively edit, expand and filter them.  We will demonstrate our
  system on two data sets, one about European mammals and climate, the
  other containing US census and election funding data.
\end{abstract}

\category{H.2.8}{Information Systems}{Database Applications}[Data Mining]
\category{H.5.2}{Information Interfaces and Presentation}{User Interfaces}

\terms{Redescription Mining, Geospatial data, Application}

%\keywords{ACM proceedings, \LaTeX, text tagging} % NOT required for Proceedings

\section{Introduction}
Finding multiple ways to characterize the same entities is an important way of gaining insight in your data.
In
redescription mining the input contains entities with two sets of
characterizing variables. The task is to find a pair of queries, one
query for both sets of variables, such that both queries describe
(almost) the same set of entities.

In biology, the bioclimatic constraints that must be met 
for a certain species to survive constitute that species' bioclimatic envelope
(or niche\footnote{The term \emph{niche} is in this paper used in Grinnellian
sense \cite{grinnell17niche}, considering only environmental variables, not 
inter-species competition or such.}),
and finding such envelopes can help, e.g.\ to predict the results of 
global warming~\cite{pearson03predicting}.
This problem correspond to f
It is here where \emph{redescription mining} comes to help.  In niche-finding, the entities
would be spatial locations, one set of variables would be the fauna
and the other set would contain the bioclimatic variables. A very
simple example of a redescription in this setting could say that the
area where polar bears live is the area where March's mean temperature
is between $-16$ and $-11$ degrees Celsius and May's mean temperature
is between $-3$ and $-7$ degrees Celsius.




\section{Redescription mining}

\note{Two views of redescription mining}
The results of redescription mining, the redescriptions, can be
approached from two points of view. On one hand, we can study the the
variables and conditions appearing in the redescriptions, giving us
valuable information about these variables. On the other hand, we can
study the support set of the redescriptions, i.e.\ the set of
observations where both queries of a redescription hold. When the data
is geospatial, that is, the observations are connected to locations, the
latter approach becomes even more important. A meaninful spatial
redescription should define meaninful areas using meaningful
queries. The goal of \Siren\ is to facilitate the analysis of the
redescriptions using both of the approaches simultaneously.

\section{Use-case scenarios}
\label{sec:scenarios}

We demonstrate the use of \Siren\ by going through a common workflow
when mining geospatial redescriptions.

\paragraph{Initial redescription mining}
Given the data, a typical first step is to use a redescription mining
algorithm to find the initial set of redescriptions. \Siren\ uses the
\ReReMi\ algorithm~\cite{galbrun11black} for this. 

When the initial set of redescriptions is mined, user can study the
intersting redescriptions plotted on a map (see
Fig.~\ref{fig:redescription}). \note{a picture of the map window} 
Uninteresting redescriptions can be manually disactivated.

\paragraph{Editing and extending a redescription}
The true strengths of \Siren\ come to play after the initial set of
redescriptions is mined. It is typical that the algorithm returns
redescriptions that the user wants to edit. For example, some
redescriptions might look unnecessarily complex, or have unnecessarily
accurate boundaries for numerical variables. User can easily select
the redescription she wants to edit, open it in a map window and start
editing. \Siren\ updates the map and important statistics (similarity,
$p$-value) about the
redescription, allowing the user to see the effects of her edits
immediately. This makes it easy for the user to see, e.g.\ if
the simplified redescription would still be acceptably accurate.

Sometimes user wants to focus only on one of the queries in the
redescription. In such cases the resulting redescription might be very
inaccurate. \Siren\ allows the user to automatically extend given
redescription, i.e.\ let the algorithm add new items to the queries to
make the redescription as accurate as possible. 

The extension mechanism of \Siren\ can also be used to explore
alternative extensions to redescriptions. The \ReReMi\ algorithm uses
a beam search, and can therefore miss some redescriptions that are not
the best extensions at some point of the search. 

Finally, many times a set of redescriptions contain the same pair of
variables. This can happen, e.g.\ when these two variables are highly
correlated. User, however, might want to see redescriptions that have
only one of these variables, not both. \Siren\ makes this simple: user
only has to select a redescription, remove the variable she does not
want to have, unselect the variable from the list of variables, and
extend the redescription. \Siren\ will not use the unselected
variables when extending or mining redescriptions.

\note{Screenshots here}

\paragraph{Filtering redundant redescriptions}
\label{sec:filt-redund-redescr}
It is common to see a set of redescriptions that cover approximately
the same area even if they have (somewhat) different set of
variables. In such cases it is important to be able to recognise and
remove redundant redescriptions, i.e.\ redescriptions that do not
convey significant new information, lest the user be overwhelmed with
the number of found redescriptions. Again, \Siren\ allows automatic
filtering of redundant redescriptions. User can either select a
redescription and ask \Siren\ to filter out all redescriptions that
are redundant with respect to the selected redescription, or \Siren\
can go throught the whole list of redescriptions filtering out all
redescriptions that are redundant with respect to some
earlier-encountered (i.e.\ better) redescription. Naturally the user
can revert the decisions made by \Siren\ whenever she wants to.

\note{Screenshots here?}

\paragraph{Outputting the results}
\label{sec:outputting-results}
Finally, \Siren\ facilitates the distribution of the results:
redescriptions can be exported in easy-to-read format \note{ToDo:
  Export human-readable redescs and LaTeX redescs perhaps too} and the
maps associated to redescriptions can be easily converted to
publication-ready files. 

\note{Screenshots here?}

\section{Related work}
\label{sec:related-work}

\note{SHOULD THIS BE PUT IN INTRO?? -Pauli}


\section{Conclusions}
This paragraph will end the body of this sample document.
Remember that you might still have Acknowledgments or
Appendices; brief samples of these
follow.  There is still the Bibliography to deal with; and
we will make a disclaimer about that here: with the exception
of the reference to the \LaTeX\ book, the citations in
this paper are to articles which have nothing to
do with the present subject and are used as
examples only.


\bibliographystyle{abbrv}
\nocite{*}
\bibliography{bibsiren}  

\balancecolumns
% That's all folks!
\end{document}

%%% Local Variables: 
%%% mode: latex
%%% TeX-master: t
%%% End: 
