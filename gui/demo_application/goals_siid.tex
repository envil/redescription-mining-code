
\section{Goals for Interactive and Visual Redescription Mining}
\label{sec:goals-inter-visu}

In this section we discuss our goals for an interactive and visual
redescription mining tool. Some of these goals are general to any
interactive and visual data mining tool (and we spend less time on
discussing why they are desirable), some are specific to
redescription mining. We divide the discussion between interaction and
visualization, though we emphasize that these goals are not independent.

\note{By no means to do I mean that we should follow some orthodox
  dichtonomy between visualization and interaction. They are
  interlinked, and that's fine. My goal is to ease the presentation by
  giving the user some high-level structure. If something is both
  visualization and interaction, we can mention it in both, putting
  emphasizis on where it belongs more.}

\subsection{Visualization of Results}
\label{sec:goals-visualization}
As a basis for our discussion, we use the taxonomy of interactions for
visual analytics proposed by Heer and
Shneiderman~\cite{Heer:2012:IDV:2133806.2133821}. The bold-face terms
correspond to their taxonomy.

\note{Filter \& sort refers to filtering and sorting what's
  visualized, right. So, yes, actions to lists of redescriptions are
  there, but also (a feature we sort-of have) selecting the geospatial
  area where we see the dots (zooming in/out) is filtering. These
  things don't need all to be grand and sexy, but they have to be.}
\note{I don't quite agree with zooming as filtering, and zooming is
  mentioned later anyways} The most fundamental goal when designing a
tool for visual data analysis is, of course, to have a good
\textbf{visualization}. With geospatial redescriptions, a map is the
most natural option. Thus our tool should be able to plot the
redescriptions on a map. But in order to effectively select the
content of the visualizations, the user needs means to \textbf{filter}
and \textbf{sort} the results mined. In the case of redescription
mining, the user should be able to sort the returned redescriptions
based on different criteria, such as accuracy, support size,
statistical significance, or query length (i.e.\ number of literals).
To some extent, filtering can be regarded as sorting with a cut-off
value. Hence, sorting should naturally use the same criteria and
similar results display as sorting. Additional criteria might affect
sorting, including the described geographical area and redundancy.
During the analysis, the user should be allowed to \textbf{derive} new
data. That is, new variables obtained by aggregating existing
variables might better capture the studied phenomenon. Hence, their
introduction during the mining process would support the
analysis. While modifying the way the information is represented,
deriving new variables is also a mean to interact with the mining
process.

\note{One can select a redescription to visualize, select an area of
  interest, a particular entity.  We do not have different levels of
  details, so, there is not much to navigating...}  \note{In addition,
  the user should be able to select a geographical area and get info
  on the data on that area. Remember the pop-up window I mentioned?
  Further levels of detail could come from whether she sees the info
  for just the variables involved in the queries, or for all
  variables.}  In order to manipulate the views, the user needs to be
able to \textbf{select} the data he wants to visualize. In the present
case, he can primarily choose a redescription to plot. Then, he can
edit the queries, modifying literals and altering the bounds of real
valued variables.  The user might need to \textbf{navigate} inside the
view, typically looking first at the redescription over the whole
area, before zooming and panning to see more details. On a high level,
the user might only be able to see whether either query hold on a
region. Focusing on particular area, he might obtain more detailed
information about the actual state of the variables and what makes a
query hold or not in a particular location, for instance by clicking
or hovering over a dot in the map. Several views and the data might
need to be \textbf{coordinated}.  Modifications made to a
redescription should be reflected immediately on the map(s). In
addition, it could be useful to allow the user to bind maps together,
so that panning and zooming are applied to all maps simultaneouly. In
that way, detailed comparison of the support of different
redescriptions would be facilitated.  Maps can be opened in detachable
tabs, to be inspected side by side or sequentially and be
\textbf{organized} using the system's or a dedicated windows tiling.

For any interactive tool, undo and redo are minimal functionalities to
allow reverting actions, making interation safe and comfortable.  The
user should be able to save the current status of the analysis
process, i.e. all current redescriptions, opened lists and maps to
punctuate the process. \textbf{Recording} the interaction history and
turning it into editable and parameterizable macros provides a mean to
repeat a sequence of actions and automate repetitive tasks.  The tool
should support \textbf{annotation} in order to keep track of the
thought path during the analysis.  For example, this could be achieved
by generating annotable screen shots of the current window of
interest, and by adding comments to the interaction history and
macros.  Organizing the history and macros into blocks would help
clarifying the logical structure of the analysis.  Furthermore, with
the ability to link to objects in the current environment such as
redescriptions, groups of entities or literals, they could be
explicitely related to each other.  Data analysis is often a
collaborative effort, involving several users. Then, \textbf{sharing}
information becomes crucial.  Easy export and import of redescriptions
lists, maps and macros, possibly with comments and annotations is a
very important feature towards that aim.  Finally, giving clear names
to the actions and providing feedback on their application helps
\textbf{guiding} users along the analysis process. Example macros with
detailed explanations, to be replayed step-by-step, represent a good
mean to introduce new users to the tool.  These latter goals pertain
automating interactions, attaching a meaning to sequences of
interactions, allowing segmented interactivity, e.g. when different
users collaborate, using the tool in turn. Hence they are also closely
tied to the interaction with the mining process, to which we now turn.

\subsection{Interaction with the Mining Process}
\label{sec:goals-interaction}

A desirable behavior for an interactive program is the production of
meaningful results at \goal{anytime}.  In other words, if the mining
process is stopped, it can nevertheless return results which are
valid, albeit possibly partial.  This is related to the possibility to
obtain preliminary results while the mining is still underway.  Such a
feature contributes to the ability of the program to respond quickly
to instructions from the users.  It is also possible to first run the
algorithm allowing only short queries, say, at most a couple of
literals on either side and let the user choose the ones that seems
promising and should be further extended. Low latency or even
\goal{instantaneity} is a core quality of an interactive tool and is
important catch and keep the user's attention. At least, the tool
should provide instant feedback about what is happening.

The automation level of the whole mining process could be adaptative.
From fully manual, where the users writes down redescriptions and the
tools simply evaluates them, to fully automated where the program
mines the list of best redescriptions using static predefined
constraints, it could also be partially automated, with the tool
suggesting best extensions at each step and asking for approval from
the user.

Consider extending an existing redescription with a real-valued
literal. Instead of a map plot based on their geographical location, a
figure where the areas are represented as colored dots plotted along
the x-axis depending on the value taken by the chosen variable would be
useful for determining the optimal interval for that variable. Indeed,
the user could observe which values occur in locations that belong to
different parts of the current redescription. Then, he could fix the
bounds in consequence, for example using sliders.  The tool could even
indicate which are the best bounds but also update the best
corresponding upper bound when the user moves the lower bound, and
vice versa. This is a prime example of instant interaction with the
mining process through apt visualization.  \note{We might need an
  illustration for that one...?}

Proper interactivity with the program also requires allowing the user
to specify constraints for the search. Possible constraints include
specifying variables or geographical areas that should be excluded
from the redescriptions or modifying the minimum acceptable accuracy.
For such constraints that constitute filtering criteria, there can be
three different degrees of integration with the algorithm, that is of
how far they are pushed into the mining process instead of applied a
posteriori.  The degree zero of integration happens when the user
manually filters the raw output. Instead, the program can
automatically filter its results before reporting. The highest degree
of integration implies incorporating such filtering criteria during
the search to avoid generating the unwanted results in the first
place.  Still, a compromise needs to be found between supporting
deep integration and accepting a broad range of constraints,
e.g. through a flexible specification framework. Indeed, these are two
 typically conflicting goals.

 More generally, the user should be able to specify interest and
 lack of it. Selecting a redescription to be
 edited and extended is a way of expressing curiosity towards the
 involved conditions or area. Similarly, he should be able to prevent
 the algorithm to search further in directions he deems uninteresting.
 One way of doing so is to merely pick out variables or locations that should be
 ignored.  Another way is to select a redescription and specify that
 results of this kind are of no interest. That could happen because the
 information provided by the redescription is known a priori by the
 analyst, on one hand, or in case it is contained in an earlier
 redescription, i.e. the latter one is redundant, on the other hand.

%%% Local Variables: 
%%% mode: latex
%%% TeX-master: "siren_iid"
%%% End: 
