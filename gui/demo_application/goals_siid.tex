
\section{Goals for Interactive and Visual Redescription Mining}
\label{sec:goals-inter-visu}

In this section we discuss our goals for an interactive and visual
redescription mining tool. Some of these goals are general to any
interactive and visual data mining tool (and we spend less time on
discussing why they are desirable), some are specific to
redescription mining. We divide the discussion between interaction and
visualization, though we emphasize that these goals are not independent.

\subsection{Goals for Visualization}
\label{sec:goals-visualization}
As a basis for our discussion, we use the taxonomy of interations for
visual analytics proposed by Heer and
Shneiderman~\cite{Heer:2012:IDV:2133806.2133821}. The bold-face terms
correspond to their taxonomy.

[Visualize is about selecting the right visual cues for presenting the data, i.e. for us colored dots on a map.
Is the redescription list a visualization?
Filter and sort is meant here within a visualization, there is no filtering or sorting on the maps views, especially as we don't show several redescriptions on the same map.
So it just boils down to sorting and filtering a list, nothing really special.
This three do not interact with the mining process, they are only about showing static results.
Deriving new variables, however, is a way to interact with the mining process.]
The most fundamental goal when designing a tool for visual data
analysis is, of course, to have a good \textbf{visualization}. With
geospatial redescriptions, a map is the most natural option. Thus our
tool should be able to plot the redescriptions on a map. But in order
to effectively select the content of the visualizations, the user
needs means to \textbf{filter} and \textbf{sort} the results mined. In
the case of redescription mining, the user should be able to sort the
redescriptions based on different criteria, such as the accuracy of
the redescription, the size of its support, its statistical
significance, or the length of the queries (i.e.\ their number of
literals).  To some extent, filtering can be regarded as sorting with
a cut-off value. Hence, sorting should naturally use the same criteria
and similar results display as sorting. Additional criteria might
affect sorting, e.g. the described geographical area and
redundancy. During the analysis, the user should be allowed to
\textbf{derive} new data. That is, new variables obtained by
aggregating existing variables might better capture the studied
phenomenon. Hence, their introduction during the mining process would
support the analysis.

[One can select a redescription to visualize, select an area of interest, a particular entity.
We do not have different levels of details, so, there is not much to navigating...]
In order to manipulate the views, the user needs to be able to
\textbf{select} the data he wants to visualize. In the present case,
he can primarily choose a redescription to plot. Then, he can edit the
queries, modifying literals and altering the bounds of real valued
variables. In addition, it can be useful to obtain information about
the state of the variables and what makes a query hold or not in a
particular location by clicking or hovering over a dot in the map.
The user might need to \textbf{navigate} inside the view, typically
looking first at the redescription over the whole area, before zooming
and panning to see more details. Several views and the data might need
to be \textbf{coordinated}.  Modifications made to a redescription
should be reflected immediately on the map(s). In addition, it could
be useful to allow the user to bind maps together, so that panning and
zooming are applied to all maps simultaneouly. In that way, detailed
comparison of the support of different redescription would be
facilitated.  Maps can be opened in detachable tabs, to be inspected
side by side or sequentially and be \textbf{organized} using the
system's windows tiling, or a dedicated one.

Undo and redo are minimal functionalities to allow reverting actions,
making interation safe and comfortable.  The user should be able to
save the current status of the analysis process, i.e. all current
redescriptions, opened lists and maps to punctuate the
process. \textbf{Recording} the interaction history and turning it
into editable and parameterizable macros provides a mean to repeat a
sequence of actions and automate repetitive tasks.  The tool should
support \textbf{annotation} in order to keep track of the thought path
during the analysis.  For example, this could be achieved by
generating annotable screen shots of the current window of interest,
and by adding comments to the interaction history and macros.
Organizing the history and macros into blocks would help clarifying
the logical structure of the analysis.  Furthermore, with the ability
to link to objects in the current environment such as redescriptions,
groups of entities or literals, they could be explicitely related to
each other.  Data analysis is often a collaborative effort, involving
several users. Then, \textbf{sharing} information becomes crucial.
Easy export and import of redescriptions lists, maps and macros,
possibly with comments and annotations is a very important feature
towards that aim.  Finally, giving clear names to the actions and
providing feedback on their application helps \textbf{guiding} users
along the analysis process. Example macros with detailed explanations,
to be replayed step-by-step, represent a good mean to introduce new
users to the tool.

\subsection{Goals for Interaction}
\label{sec:goals-interaction}
\note{Should we use interactivity here, as a general succession of interactions?}
\note{just some buzzwords now}

Any-time

Instant editing of redescriptions.

When editing a real-valued literal, show a scatter plot of the
entities based on the values of the variable considered, with colors
indicating the current status of the rest of the query (possibly
scatter using two variables). This could help the user fix the bounds,
using sliders for example.  Indicate which are the best bounds and the
best corresponding upper bound when the user moves the lower slider,
and vice versa.  This is a type of visualization too, which could be
useful for non geographical data too.

Pushing filtering into the mining process instead of a
posteriori. (Three levels: manually filter, automatically filter
results before reporting, filtering during the search to avoid
generating the result in the first place.)

Automated extension

The level of automation of the whole mining process could be adaptative: from fully manual (users write down redescriptions and the tools simply evaluates them), partialy automated (the tool suggests best extension at each step and ask for approval from the user), to fully automated (returns the list of best redescriptions using static predefined constraints).

Exclusion of variables/geographical areas

It's important to keep track of the constraints used when finding a
redescription. (cf. pitfalls) It can get intricate if some constraints
are modified during the extension process to redirect the search, like, after first literal is appened, try to exclude that area. Interpretation of the results becomes fairly difficult...

$\ldots$

%%% Local Variables: 
%%% mode: latex
%%% TeX-master: "siren_iid"
%%% End: 
