\section{The Algorithms}
\label{sec:algorithms}
\note{All explanations about the workings of ReReMi \& Siren here}

\Siren\ is an interactive tool for mining and visualizing geospatial
redescriptions. At its core is the \ReReMi\ redescription mining
algorithm~\cite{galbrun11black,galbrun12black}. 

This greedy algorithm
uses an efficient on-the-fly discretization technique to extend
redescription mining to categorical and numerical variables.
It considers the queries over such variables that can be parsed in linear
order, without trees, with every variable allowed to appear only once.
They constitute a subset of Boolean formulae that
provides a good compromise between expressive power, difficulty of the
search, and interpretability.

Yet, the search space remains exponential and we still resort to a
heuristic pruning during the search.  We use a strategy similar to
beam-search to explore the solution space.  The basic idea is to
construct queries bottom-up, starting from singleton redescriptions
(i.e.\ both queries contain only one literal) and progressively
extending them by appending operators and
literals. % For example, we could start with a pair $(a,
% \lnot b)$, and try to extend it to $(a\land c, \lnot b)$, $(a \lor c,
% \lnot b)$, $(a \land \lnot c, \lnot b)$, etc.
After evaluating all possible one-step extensions, we select the best
candidates and extend them in turn. This process stops when no new
redescription can be generated.

We exploit some simple observations to make the computation of
accuracy more efficient. This allows to evaluate candidates faster,
which is particularly important for an interactive tool.  

We compute a
\pValue{} that represents the probability that two random queries with
marginal probabilities (i.e.\ the fraction of entities supporting
them) equal to those of $q_\iLHS$ and $q_\iRHS$ have an intersection
equal to or larger than $\abs{\supp(q_\iLHS, q_\iRHS)}$. This
probability uses the binomial distribution. The higher the \pValue,
the more likely it is to observe such a support for independent
queries, and the less significant the query. Redescriptions with too
high \pValue{} can be filtered out.

Threading is employed to run mining tasks in the background. This preserves the tool's
interactivity while the communication is maintained to provide feedback
about the ongoing mining.

\note{This had been commented out... We need to say these things don't we?}
\Siren\ and \ReReMi\ are implemented in Python.  The interface is
built with the \texttt{wxPython} Open Source GUI toolkit, ensuring
cross-platform compatibility. 
 The \texttt{matplotlib} library enables
to generate high quality figures, seamlessly integrated in the
interface.  \Siren\ allows for simple editing of the redescriptions
thanks to flexible parsing of different representations. It can handle
any data provided in a compatible format.  

\section{How the Algorithms Meet the Goals}
\label{sec:scenarios}

\note{The old use-case scenario text is a good basis for this, but
  needs to be re-worked a lot.\\
Do we want to talk only about the functionalities that are already implemented, can be reasonably implemented in a short delay, or any wishable functionality? And insert links to the goal section... 
}

We exemplify the usage of \Siren\ by going through a generic work-flow
of mining geospatial redescriptions, detailing typical steps in the
process.  This specific example concerns the application of \Siren\ on
the task of bioclimatic niche finding using data that describes
spatial areas of Europe, squares of side roughly 50 kilometers.  The
left hand side data contains information about the mammals that live
in these areas, while the right hand side consists of bioclimatic
variables\footnote{The data comes from two publicly available
  datasets: European mammal atlas~\cite{mitchell-jones99atlas} and
  Worldclim climate data~\cite{hijmans05very}.}.  Nonetheless, \Siren\
is a flexible tool that can be used with different datasets from
various application domains.

% A screenshot of the system,
% displaying a list of redescriptions with one
% particular redescription plotted on a map is shown in
% Figure~\ref{fig:both_panels}.

\prg{Initial redescription mining} A natural starting point for the
analysis of any given data is to use a redescription mining algorithm
to find an initial set of redescriptions.  This can be done withing
\Siren{} by running the extension mechanism on an empty redescription.

Following the principle of first providing an overview of the
results then focusing on specific items.
The redescriptions found are presented as a list, which extends as the
mining process progresses and supports sorting on various criteria.
Then, the user can select a redescription of his choice and visualize
it on a map.

\prg{Extending a redescription} Sometimes the user wants to focus only
on one of the queries, on some particular variable of interest or on a
part of an existing redescription.  \Siren\ allows the user to
automatically extend a given redescription, i.e.\ let the algorithm
add new literals to the queries to make the redescription as accurate
as possible.
% (see Fig.~\ref{fig:extending}). 

The extension mechanism of \Siren\ is based on the beam search
implemented in the \ReReMi\ algorithm~\cite{galbrun11black}. In this case, the intermediate redescriptions
explored during the search are returned at each step, allowing to study
more specific alternative extensions to a redescription that were discarded from the beam
because they were not among the best extensions at some point of the search.

In the climatic niche-finding task, for instance, we might select a
species, say, the Southwestern Water Vole and look for best extensions
starting from that single variable.  Returned extensions can be
visualized side by side and compared as shown in
Figure~\ref{fig:comparison}. Here, the best found extension has
accuracy $0.665$ (per Jaccard coefficient):
\begin{equation*}
%\footnotesize
\begin{array}{l}
\text{Southwestern Water Vole }\lor\text{ Gray Dwarf Hamster }\lor\text{ Savi's Pine Vole }\\[1mm]
\quad\lor\text{ Mediterranean Monk Seal}\\[3mm]
[11.2 \leq t_{3}^{+}] \land  [0.51 \leq t_{1}^{=} \leq 11.333]\land  [42.75 \leq p_{10}^{=} \leq 131.81] \\[1mm]
\quad\land [50.556 \leq p_{11}^{=} \leq 176.75],
\end{array}
\end{equation*}

This redescription indicates that areas where any of the four species
lives correspond to areas where the maximum temperature in March is
above $11.2$ degrees Celsius, the average temperature in January
between $0.51$ and $11.333$ degrees Celsius and the average
precipitations in October and November range from $42.75$ to $131.81$
millimeters and from $50.556$ to $176.75$ millimeters, respectively.

\prg{Editing a redescription} It is typical that the user wants to
edit some of the obtained redescriptions. For example, some results
might be overly complex, or have exceedingly precise boundaries for
numerical variables. The user can easily select a redescription to
modify, open it in a map panel and edit it. Boundaries can be altered,
literals added or removed. \Siren\ updates the map and important
statistics (accuracy, $p$-value, etc.) of the redescription, allowing
the user to see the effects of the modifications immediately and
verify, e.g.\ whether the new redescription would still be acceptably
accurate.

Continuing with our example above, we might want to reduce the
precision of the climatic constraints to integers. We could edit the
query as follows:
\begin{equation*}
%\footnotesize
\begin{array}{l}
[11 \leq t_{3}^{+}] \land  [0 \leq t_{1}^{=} \leq 12]%\\[1mm]
%\quad
\land  [42 \leq p_{10}^{=} \leq 132] \land [50 \leq p_{11}^{=} \leq 177],
\end{array}
\end{equation*}
and obtain a redescription of slightly decreased accuracy. % of $0.659$.

\prg{Using subsets of variables} 
\Siren\ allows the user to specify variables to temporarily 
avoid when extending or mining redescriptions.

For example, when two variables are highly
correlated, several redescriptions might contain them
both. The user, however, might want to consider
redescriptions with only one of these variables, not
both. \Siren\ makes that simple: the user only has to select a
redescription, remove the unwanted variable from the
redescription and unselect it from the list of variables, then extend the
redescription again. 

Alternatively, in our running example, we might want to force the
algorithm to search alternative redescriptions that do not involve any
 precipitation. For that purpose, we simply unselect all such
variables before running the extension anew. We will obtain the best
extensions containing only temperatures in the bioclimatic query, such
as the following redescription of accuracy $0.653$:
\begin{equation*}
%\footnotesize
\begin{array}{l}
\text{Southwestern Water Vole }\lor\text{ Cape Hare }\lor\text{ Savi's Pine Vole }\\[1mm]
\quad\lor\text{ Mediterranean Monk Seal}\\[3mm]
( [11.2 \leq t_{3}^{+}] \land  [20.1 \leq t_{7}^{+} \leq 32.9] %\\[1mm] 
%\quad 
\land  [0.51 \leq t_{1}^{=} \leq 11.333]) \lor  [34.0 \leq t_{8}^{+}].
\end{array}
\end{equation*}

Note that this redescription was not returned previously since the
beam search focused on better ones involving precipitation variables.

\prg{Filtering redundant redescriptions}
\label{sec:filt-redund-redescr}
It is common to see a set of redescriptions that cover approximately
the same area even if they have (somewhat) different sets of
variables.  Indeed, redescriptions belong to the family of local
patterns, with each individual pattern independently describing a
subset of the data. Mining local patterns typically returns redundant
results that require filtering.  In such cases, it is important to be
able to recognize and remove redundant redescriptions, i.e.\
redescriptions that do not convey significant new information, lest
the user be overwhelmed with the number of found
redescriptions. Again, \Siren\ allows automatic filtering of redundant
redescriptions. The user can select a redescription and ask
\Siren\ either to filter out all redescriptions that are redundant
with respect to the selected one, or to go through the whole list of
redescriptions filtering out all redescriptions that are redundant
with respect to some earlier-encountered (i.e.\ better)
redescription. Naturally, the decisions made by
\Siren\ can be reverted whenever the user wishes to.

For instance, the results returned during the extension
mentioned previously may contain many redundant redescriptions found
at different steps. We can easily sort them, e.g.\ by accuracy, select
one of interest and filter all the following results redundant with respect to it.

\prg{Outputting the results}
\label{sec:outputting-results}
Finally, \Siren\ facilitates the distribution of the results:
redescriptions can be exported in easy-to-read format and the
maps associated to redescriptions can be readily converted to
publication-ready graphics. 



%%% Local Variables: 
%%% mode: latex
%%% TeX-master: "siren_iid"
%%% End: 
