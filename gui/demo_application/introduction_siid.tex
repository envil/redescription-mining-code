\section{Introduction}
\note{TODO: start visual and interactive data mining more generally?}

Finding multiple ways to characterize the same entities is a problem
that appears in many areas of science.  In medical sciences, one might
want to find a subset of patients sharing similar symptoms and similar
genes. Describing geographical regions in terms of both their
bioclimatic conditions and the fauna that inhabits them is another
example.  A simple example of a redescription in this setting could
state that areas where Moose live are areas where February's maximum
temperature is between $-10$ and $0$ degrees Celsius and July's
maximum temperature between $12$ and $25$ degrees Celsius.
%This is actually the redescription shown in the foreground panel of
%Figure~\ref{fig:both_panels}.
\note{maybe we do want to have some screenshot already here, maybe not...}

The results of redescription mining, the redescriptions, can be
approached from two points of view. On one hand, by considering the
variables and conditions appearing in the queries, which provide
valuable information in themselves; on the other hand, by studying the
support set of the redescriptions, i.e.\ the subset of entities where
both queries of a redescription hold. 
 
\note{This explanation is unclear about split between visualization and interactivity and dependance between two.}
To analyse the
redescriptions, the ability to visualize the support sets is very
helpful. With geospatial data, the visualization is rather simple:
plot the support sets in a map. But just plotting the results on a map
is not enough: the user must also be able to interact with the
program. This interaction can be conceptually divided in two
sub-phases: interacting with the data mining algorithm and interacting
with the result visualization. A third level of interaction happens
between these two conceptual phases: the user moves back-and-forth
between issuing commands to find new results and examining the
already-found results. We argue that a good interactive data mining
tool should facilitate all three types of interaction. In this paper
we discuss about ways to facilitate this interaction in the process of
mining redescriptions. We then present a pair of algorithms, \ReReMi\
and \Siren, and explain how they implement interactivity and
visualization for redescription mining. Lastly, we discuss some
possible pitfalls associated with interactive, visual redescription
mining. But first, we formally define the redescription mining
problem.

%%% Local Variables: 
%%% mode: latex
%%% TeX-master: "siren_iid"
%%% End: 
